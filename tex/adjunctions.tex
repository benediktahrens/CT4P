\chapter{Adjunctions}
\label{sec:adjunctions}

\begin{dfn} A pair $(F,G)$ of functors $F : \CC\to\DD, G:\DD\to\CC$ is called an \textbf{adjoint pair} if for every objects $X\in\Ob{\CC}$ and $Y\in\Ob{\DD}$, there exists a bijection 
\[
\alpha_{X,Y} : \CHom{\DD}{F(X)}{Y} \to \CHom{\CC}{X}{G(Y)},
\]
which are moreover natural in both $X$ and $Y$, i.e. for each $f\in\CHom{\CC}{X_1}{X_2}$ and $g\in\CHom{\DD}{Y_1}{Y_2}$, the following diagrams commute:
\begin{eqnarray}
\begin{tikzcd}
\CHom{\DD}{F(X_2)}{Y} \arrow[r, "\alpha_{X_2,Y}"] \arrow[d,swap, "\co{F(f)}{-}"] & \CHom{\CC}{X_2}{G(Y)} \arrow[d, "\co{f}{-}"] \\
\CHom{\DD}{F(X_1)}{Y} \arrow[r,swap, "\alpha_{X_1,Y}"] & \CHom{\CC}{X_1}{G(Y)}
\end{tikzcd}\\
\begin{tikzcd}
\CHom{\DD}{F(X)}{Y_1} \arrow[r, "\alpha_{X,Y_1}"] \arrow[d,swap, "\co{-}{g}"] & \CHom{\CC}{X}{G(Y_1)} \arrow[d, "\co{-}{G(g)}"] \\
\CHom{\DD}{F(X)}{Y_2} \arrow[r,swap, "\alpha_{X,Y_2}"] & \CHom{\CC}{X}{G(Y_2)}
\end{tikzcd}
\end{eqnarray}
If $(F,G)$ is an adjoint pair, we call $F$ the \textbf{left adjoint} of $G$ and we call $G$ the \textbf{right adjoint} of $F$ and we denote $F \dashv G$.
\end{dfn}

\begin{exer} Let $F:\CC\to\DD$ be a functor. Show that if $F$ has a left (resp. right) adjoint $G$, then $G$ must be unique up to isomorphism\footnote{Isomorphism w.r.t the functor category.}.
\end{exer}

\begin{exer} Let $U:\MON\to \SET$ be the forgetful functor (defined in \cref{example:forgetful_montoset}) which maps a monoid to its underlying set and let $F : \SET\to\MON$ be the functor which maps a set to the free monoid of this set (defined in \cref{exa:freemonoids}). Then is $(F,U)$ an adjoint pair. (Hint: Use \cref{prop:UVP_forget_montoset}). 
\end{exer}

\begin{exer}\label{exer:adjunction_homtensor_currying} Let $Y$ be a set. Show that the functor (induced by the following mapping on objects)
\[
- \times Y : \SET\to \SET: X\mapsto X\times Y,
\]
has a right adjoint, which is given by the functor (induced by the following mapping on objects)
\[
\CHom{\SET}{Y}{-} : \SET\to\SET : X\mapsto \CHom{\SET}{Y}{X}.
\]
Is there an analogous statement in the category $\COQ$ instead of $\SET$?
\end{exer}

\begin{thm} Let $(F,G)$ be a pair of functors $F : \CC\to\DD, G: \DD\to\CC$. The following statements are equivalent:
\begin{enumerate}
\item $(F,G)$ is an adjoint pair.
\item There exists natural transformations 
\[
\NatTrans{\eta}{\Id[\CC]}{\co{F}{G}}, \quad \NatTrans{\epsilon}{\co{G}{F}}{\Id[\DD]},
\]
such that for all $X\in\CC$ and $Y\in\DD$ the following diagrams commute:
\[
\begin{tikzcd}
F(X) \arrow[r,"{F(\eta_X)}"] \arrow[rd,swap, "{\Id[F(X)]}"] & F(G(F(X))) \arrow[d, "{\epsilon_{F(X)}}"] \\
& F(X)
\end{tikzcd} \quad 
\begin{tikzcd}
G(Y) \arrow[r,"{\eta_{G(Y)}}"] \arrow[rd,swap, "{\Id[G(Y)]}"] & G(F(G(Y))) \arrow[d, "{G(\epsilon_{Y})}"] \\
& G(Y)
\end{tikzcd}
\]
\end{enumerate}
In case $(F,G)$ satisfies these (equivalent) conditions, we call $\eta$ the \textbf{unit of the adjunction} and $\epsilon$ the \textbf{counit of the adjunction}. The equalities in condition $2$ are called the \textbf{triangle identities}.
\begin{proof}
First, assume that $(F,G)$ is an adjoint pair. We have to define the unit and counit and show that the triangle identities hold:
\begin{itemize}
\item \textbf{Unit}: For each $X\in\Ob{\CC}$, we should first define $\eta_X \in \CHom{\CC}{X}{G(F(X))}$. Since $F \dashv G$, we have a bijection
\[
\alpha_{X,FX} : \CHom{\DD}{FX}{FX} \to \CHom{\CC}{X}{G(F(X))},
\]
hence, we define $\eta_X := \alpha_{X,FX}(\Id[FX])$. We now show that $(\eta_X)_{X\in\Ob{\CC}}$ forms a natural transformation: Assume $f\in\CHom{\CC}{X}{Y}$. We have to show that the following diagram commutes:
\[
\begin{tikzcd}
X \arrow[rr, "{\alpha_{X,FX}(\Id[FX])}"] \arrow[d,swap, "f"] && G(F(X)) \arrow[d, "G(F(f))"] \\
Y \arrow[rr, swap, "{\alpha_{Y,FY}(\Id[FY])}"] && G(F(Y))
\end{tikzcd}
\]
That this square is indeed commutative follows from the following computation:
\begin{equation}\label{eqn:unitnaturality_fromhomsetadj}
\co{f}{\alpha(\Id[FY])} = \alpha(\co{F(f)}{\Id[FY]}) = \alpha(F(f)) =  \alpha(\co{\Id[FX]}{F(f)}) = \co{\alpha(\Id[Fx])}{G(F(f))},
\end{equation}
where the first and last equality hold by naturality of $\alpha$.
\item \textbf{Counit}: For each $Y\in\Ob{\DD}$, we the counit $\epsilon_Y \in \CHom{\DD}{F(G(Y))}{Y}$ is defined as the image of $\Id[G(Y)]$ of the bijection
\[
\alpha^{-1}_{GY,Y} : \CHom{\CC}{GY}{GY}\to \CHom{\DD}{F(G(Y))}{Y}.
\]
That $\epsilon$ indeed forms a natural transformation is analogous to the computation in \cref{eqn:unitnaturality_fromhomsetadj}.
\item \textbf{Triangle identities}: Both the triangle identities are proved analogously, hence we will only show the first triangle identity, i.e. $\Id[FX] = \co{F(\eta_X)}{\epsilon_{FX}}$. Unfolding the definition of $\epsilon$, this is equivalent to showing:
\[
\Id[FX] = \co{F(\eta_X)}{\alpha^{-1}_{GFX,FX}(\Id[GF(X)])}.
\]
Since the components of $\alpha$ are bijections, this is equivalent to showing 
\[
\alpha(\Id[FX]) = \alpha(\co{F(\eta_X)}{\alpha^{-1}_{GFX,FX}(\Id[GF(X)])})
\]
This indeed holds by the following computation:
\[
\alpha(\co{F(\eta_X)}{\alpha^{-1}_{GFX,FX}(\Id[GF(X)])}) = \co{\eta_X}{\alpha(\alpha^{-1}(\Id[GFX]))} = \co{\eta_X}{\Id[GFX]} = \eta_X,
\]
where the first (resp. second) equality holds by naturality (resp. bijectiveness) of $\alpha$.
\end{itemize}
This concludes the proof of $(1)\implies (2)$. Now assume that $(2)$ holds. We have to construct bijections
\[
\alpha_{X,Y} : \CHom{\DD}{F(X)}{Y} \to \CHom{\CC}{X}{G(Y)},
\]
which are natural in $X$ and $Y$. Let $g\in \CHom{\DD}{F(X)}{Y}$. Define $\alpha_{X,Y}(g)$ as the composite:
\[
X \xrightarrow{\eta_X} G(F(X)) \xrightarrow{G(g)} G(Y).
\]
For the other direction, let $f\in \CHom{\CC}{X}{G(Y)}$. Define $\alpha^{-1}_{X,Y}(f)$ as the composite:
\[
FX \xrightarrow{F(f)} F(G(Y)) \xrightarrow{\epsilon_Y} Y.
\]
That $\alpha$ and $\alpha^{-1}$ are inverses of each other follows from the following computation:
\begin{eqnarray*}
\alpha(\alpha^{-1}(f)) = \alpha(\co{F(f)}{\epsilon_Y}) =& \co{\eta_X}{G(\co{F(f)}{\epsilon_Y})} &\text{ by definition, }\\
	=& \co{\eta_X}{(\co{GF(f)}{G\epsilon_Y})} &\text{ by functoriality of $G$},\\
	=& \co{(\co{\eta_X}{GF(f)})}{G\epsilon_Y} &\text{ by associativity},\\
	=& \co{(\co{f}{\eta_{GY})}}{G\epsilon_Y} &\text{ by naturality of $\eta$}\\
	=& \co{f}{(\co{\eta_{GY}}{G\epsilon_Y})} &\text{ by associativity}\\
	=& f &\text{by triangle identity}.
\end{eqnarray*}
The other equality is shown analogous by using functoriality of $F$, naturality of $\epsilon$ and the other triangle identity.

It remains to show the naturality of $\alpha$ in both $x$ and $y$ which is left to the reader as a good exercise on diagram chasing. %\KW{TODO: Give the proof of the naturality}
\end{proof}
\end{thm}
%%% Local Variables:
%%% mode: latex
%%% TeX-master: "CT4P"
%%% End:
