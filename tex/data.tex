
\chapter{Inductive Datatypes and Initial Algebras}
\label{sec:initial-algs}

In this chapter, we discuss how inductive datatypes as used in functional programming give rise to initial objects in suitable categories.

We first look at some example datatypes in \cref{sec:examples}.
Afterwards, we study a general framework that allows us to discuss many datatypes at once, in \cref{sec:datatypes-as-initial}.

\section{Examples}
\label{sec:examples}

\begin{exer}\label{exer:nat-initial}
  Consider the datatype
\begin{lstlisting}[mathescape=true]
data $\NN$ ::=
| $\Zero$ : $\NN$
| $\Succ$ : $\NN \to \NN$
\end{lstlisting}
%
%
This inductive datatype comes with a recursion principle to define functions from the natural numbers to any other type $X$,
\[
  \binaryRule
  {z \in X}
  {s : X \to X}
  {\rec(z,s) : \NN \to X}
  {rec}
\]
such that
\begin{equation}\label{eq:computation-rules-nat}
  \rec(z,s)(\Zero) = z \quad\text{ and }\quad \rec(z,s)(\Succ(n)) = s (\rec(z,s)(n)).
\end{equation}
Furthermore, it satisfies the following induction principle:
\[
  \binaryRule
  {P(\Zero)}
  {\forall n : \NN, P(n) \Rightarrow P(\Succ (n))}
  {\forall n : \NN, P(n)}
  {ind}
\]

and the following  category:
\begin{itemize}
\item Objects are triples $(X, z \in X, s : X \to X)$ with $X$ a set;
\item Morphisms from $(X, z \in X, s : X \to X)$ to $(X', z' \in X', s' : X' \to X')$ are functions
  $f : X \to X'$ such that the following diagrams commute:
  \[
    \begin{tikzcd}
      1 \ar[r, "z"] \ar[rd, "z'"']
      &
      X \ar[d, "f"]
      \\
      &
      X'
    \end{tikzcd}
    \quad
    \begin{tikzcd}
      X \ar[r, "s"] \ar[d, "f"']
      &
      X \ar[d, "f"]
      \\
      X' \ar[r, "s'"]
      &
      X'
    \end{tikzcd}
  \]
\item Composition and identity are given by composition of functions in $\SET$.
  (Check that this is well-defined, that is, that the composition of two functions making the above diagrams commute makes the right diagrams commute again.)
\end{itemize}

We now show that the recursion and induction principles can be used to show that the natural numbers, with $\Zero$ and $\Succ$, are an initial object in this category.

\begin{enumerate}
\item Draw the equations of \cref{eq:computation-rules-nat} as diagrams in the category of sets.
\item Show that the triple $(\NN, \Zero, \Succ)$ is an initial object in this category.
\end{enumerate}

\end{exer}

The next exercise is very similar to \cref{exer:nat-initial}.

\begin{exer}\label{exer:aexp}
  Consider the datatype
\begin{lstlisting}[mathescape=true]
data Exp ::=
| Int     : $\mathbb{Z}$ $\to$ Exp
| Plus    : Exp $\times$ Exp $\to$ Exp
| Squared : Exp $\to$ Exp
\end{lstlisting}
  and consider the following category:
  \begin{itemize}
  \item Objects are quadruples $(X, I : \mathbb{Z} \to X, P : X \to X \to X, S : X \to X)$ with $X$ a set/type;
  \item Morphisms from $(X, I, P, S)$ to $(X', I', P', S')$ are functions
    $f : X \to X'$ such that the following diagrams commute:
    \[
      \begin{tikzcd}
        \mathbb{Z} \ar[r, "I"] \ar[rd, "I'"']
        &
        X \ar[d, "f"]
        \\
        &
        X'
      \end{tikzcd}
      \quad
      \begin{tikzcd}
        X\times X \ar[r, "P"] \ar[d, "f \times f"']
        &
        X \ar[d, "f"]
        \\
        X' \times X' \ar[r, "P'"]
        &
        X'
      \end{tikzcd}
      \quad
      \begin{tikzcd}
        X \ar[r, "S"] \ar[d, "f"']
        &
        X \ar[d, "f"]
        \\
        X' \ar[r, "S'"]
        &
        X'
      \end{tikzcd}
    \]
  \item Composition and identity are given by composition of functions in $\SET$. (Check that this is well-defined.)
  \end{itemize}

We show that the datatype of expressions, together with its constructors, forms an initial object in the category defined above.

\begin{enumerate}
\item Write down the recursion and induction principles for |Exp|.
\item Write down the computation rules (the analogs of \cref{eq:computation-rules-nat}) for |Exp|, and draw them as diagrams.
\item Show that the quadruple consisting of the type |Exp| together with the functions |Int|, |Plus|, and |Squared|, is an initial object in this category. 
\end{enumerate}
  
\end{exer}

\begin{exer}
  Define an ``evaluation'' function $\mathsf{ev} : \mathsf{Exp} \to \mathbb{Z}$ that associates to any expression its ``semantics'' or ``denotation'', that is, the integer the expression denotes.
  Specifically, define this function as a morphism in the category of \cref{exer:aexp}.
\end{exer}

\begin{exer}[\cref{sol:natlist_is_initial}]\label{exer:natlist_is_initial}
  Consider the datatype
\begin{lstlisting}[mathescape=true]
  data NatList ::=
  | nil     : NatList
  | cons    : $\NN$ $\to$ NatList $\to$ Natlist
\end{lstlisting}

  \begin{enumerate}
  \item Write down the recursion and induction principles for |Natlist|.
  \item  Write down the computation rules (the analogs of \cref{eq:computation-rules-nat}) for |NatList|, and draw them as diagrams.
  \item Construct a suitable category $\Cat{L}$ and show that ($\mathsf{NatList}, \nil, \cons$) is initial in that category.
  \end{enumerate}
  
\end{exer}

\section{Datatypes as Initial Algebras}
\label{sec:datatypes-as-initial}



\begin{reading*}
  This chapter is strongly inspired by Varmo Vene's Ph.D.\ thesis \cite[Chapter 2]{vene_phd}.

  A good explanation of recursion on lists is given in Graham Hutton's tutorial paper \cite{DBLP:journals/jfp/Hutton99}.

  The tutorial on (co)algebras and (co)induction by Jacobs and Rutten \cite{jacobs-rutten-tutorial} provides an excellent overview to the categorical view on inductive and coinductive datatypes.
\end{reading*}

In this section we introduce (initial) algebras which allows us to define inductive data types.

\begin{dfn} Let $F:\CC\to \CC$ be an endofunctor. An \textbf{$F$-algebra} consists of the following data:
\begin{enumerate}
\item An object $X\in \Ob \CC$.
\item A morphism $\phi \in \CHom{\CC}{F(X)}{X}$.
\end{enumerate}
\end{dfn}

\begin{intu}
An algebra is roughly a set equipped with some operations, such as multiplication.
The \emph{arities}, that is, the inputs, of the operations are determined by the functor $F$.
An important class of functors are \emph{polynomial functors} built using the coproduct $(+)$ and the product $(\times)$.
Intuitively, the different summands of a polynomial functor each correspond to one datatype constructor, whereas the use of the product indicates that a constructor takes several inputs.
\end{intu}

\begin{exa}\label{exa:nno_initial_alg_maybe}
 Let $\Maybe : \SET \to \SET$ be the endofunctor given on objects by $\Maybe(X) := 1 + X$. 
 A $\Maybe$-algebra is a pair $(X,\phi)$ of a set $X$ and a function $\phi : 1 + X \to X$.
 By the universal property of the coproduct, $\phi$ is given, equivalently,
 by two functions $z$ and $s$ as follows.
 \begin{align*}
    z : 1 &\to X
    \\
    s : X &\to X
 \end{align*}
Here, think of ``$z$'' standing for ``zero'', and ``$s$'' standing for ``successor''.
\end{exa}


\begin{exa}\label{example:algebra_of_monoids} Let $F$ be the endofunctor induced by:
\begin{align*}
  F: \SET &\to \SET
  \\
  X &\mapsto 1 + (X\times X).
\end{align*}
An $F$-algebra consists of a set $X\in\SET$ together with a function $\phi:\CHom{\SET}{1 + (X\times X)}{X}$. Since $\phi$ is function from the disjoint union, we have that $\phi$ corresponds uniquely to two functions:
\begin{align*}
e &: 1\to X
\\
m &: X\times X\to X.
\end{align*}
This is precisely the data of a monoid.

Conversely, if $(M,m,e)$ is a monoid, then this induces a function
\[
1 + (M\times M) \xrightarrow{\phi = [\phi_e,\phi_m]} M,
\]
defined by pattern matching as follows:
\begin{align*}
\phi_e : 1 &\to M
\\
\star &\mapsto e
\\
\phi_m : M\times M &\to M 
\\
(x,y) &\mapsto m(x,y)
\end{align*}

The pair $(M, \phi)$ is an $F$-algebra.

\end{exa}

\begin{rem}
 Note that a monoid $(M,m,e)$ also satisfies some laws.
 The laws are not expressed in \cref{example:algebra_of_monoids}.
 To incorporate the laws, one studies instead algebras of a monad.
\end{rem}


\begin{dfn}\label{dfn:alg-hom}
Let $F:\CC\to \CC$ be an endofunctor and $(X,\phi)$ and $(Y,\psi)$ be $F$-algebras. 
A \textbf{($F$-algebra) homomorphism} from $(X,\phi)$ to $(Y,\psi)$ is a morphism $f\in \CHom{\CC}{X}{Y}$ such that the following diagram commutes:
\[
\begin{tikzcd}
FX
\arrow[r, "\phi"] 
\arrow[d,swap, "Ff"]
& X
\arrow[d, "f"] 
\\
FY
\arrow[r, swap, "\psi"] 
& Y
\end{tikzcd}
\]
\end{dfn}



\begin{exer} 
  Let $F$ be the endofunctor defined as in \cref{example:algebra_of_monoids}, i.e., the endofunctor whose algebras correspond to monoids. 
  Describe the $F$-algebra homomorphisms.
\end{exer}

\begin{dfn}\label{definition:category_of_Falgebras} Let $F:\CC\to \CC$ be an endofunctor. The \textbf{category of $F$-algebras}, denoted by $\ALG{F}$, is defined by the following data:
\begin{itemize}
\item The objects are the $F$-algebras.
\item The morphisms are the $F$-algebra homomorphisms.
\item The identity on $(X,\phi)$ is given by the identity $\Id[X]$ in $\CC$.
\item The composition is given by the composition of morphisms in $\CC$.
\end{itemize}
\end{dfn}

\begin{exer}
  \begin{enumerate}
  \item[]
  \item Draw two diagrams to illustrate \cref{definition:category_of_Falgebras}.
  \item Show that $\ALG{F}$ satisfies the properties of a category.
  \end{enumerate}
\end{exer}

We are interested in \textbf{initial objects of $\ALG{F}$}, if they exist.
We call these ``initial $F$-algebras''.
For a general endofunctor $F$, an initial $F$-algebra does not exist;
but for many interesting choices of $F$, such an initial object does exist.
Before coming to the general definition (see \cref{dfn:initial-alg}),
we consider an example.


\begin{exer}
  Consider the functor $\Maybe : \SET \to \SET$.
  \begin{enumerate}
  \item Show that the initial $\Maybe$-algebra is given by the pair $(\NN, [\Zero,\Succ])$, 
    where $\NN$ is the set of natural numbers, and $\Zero : 1 \to \NN$ and $\Succ : \NN\to\NN$ 
    are the function picking out zero and the successor function, respectively.
  \item Given any other $\Maybe$-algebra $(X,[z,s])$, unpack what it means for the square of \cref{dfn:alg-hom} to commute.
  \item Compare the data from the previous exercise to a definition of a function $f : \NN \to X$ by pattern matching (e.g., in Haskell).
  \end{enumerate}
\end{exer}

\begin{dfn}\label{dfn:initial-alg}
  Let $F:\CC\to\CC$ be an endofunctor. An \textbf{initial $F$-algebra} (if it exists) is an initial object in $\ALG{F}$.
  Unfolding the definition, this means that it is an $F$-algebra $(\Initalg{F}, \In)$ such that for any $F$-algebra $(X,\phi)$, there exists a unique morphism $\catam{\phi} \in \CHom{\CC}{\Initalg{F}}{X}$ such that the following diagram commutes:
\begin{equation}\label{eq:initial-f-alg}
\begin{tikzcd}
{F\Initalg{F}} 
\arrow[r, "\In"] 
\arrow[d,swap, "F\catam{\phi}"]
& {\Initalg{F}} 
\arrow[d, "\catam{\phi}"] 
\\
FX
\arrow[r, swap, "\phi"] 
& X
\end{tikzcd}
\end{equation}
The morphism $\catam{\phi}$ is called the \emph{catamorphism} generated by $\phi$.
\end{dfn}

\begin{exer}[\cref{sol:in_catamorphism_id}]\label{exer:in_catamorphism_id}
  Let $F:\CC\to\CC$ be an endofunctor and let $(\Initalg F, \In)$ be an initial algebra.
  Show that
  \[\catam{\In} = \Id[\Initalg{F}].\]
\end{exer}

\begin{exer}\label{exer:bool_as_initial_algebra}
  Let $\mathbf{Bool}$ be the inductive data type generated by the following two constructors:
\begin{lstlisting}
    True  : Bool
    False : Bool
\end{lstlisting}
  Define an endofunctor $F:\SET\to \SET$ such that the $F$-algebras can be described as triples $(X, b_1, b_2)$ with $X$ a set and $b_1,b_2\in X$.
  
  Moreover, show that $(\mathsf{Bool}, \mathsf{True}, \mathsf{False})$ is an initial object in $\ALG{F}$.
\end{exer}

\begin{exer}\label{exer:coproduct_as_initial_algebra}
  The disjoint union (i.e., the coproduct) of two sets $X$ and $Y$ can also be described as an inductive data type; indeed, it is generated by the following two constructors:
\begin{lstlisting}
    f : X -> X+Y
    g : Y -> X+Y
\end{lstlisting}
  Define an endofunctor $F:\SET\to \SET$ such that the $F$-algebras can be described as triples $(C, i_l, i_r)$ with $C$ a set and $i_l : X\to C, i_r : Y\to C$ be functions.
 
  Moreover, show that $(X + Y, \inl, \inr)$ is the initial object in $\ALG{F}$, where $X+Y$ is the disjoint union of $X$ and $Y$ (i.e. the coproduct in $\SET$) and $\inl:X\to X+Y, \inr:Y\to X+Y$ the canonical inclusions.
\end{exer}

\begin{exer}\label{exer:conatural_numbers_are_not_initial}
Let $\mathbb{N}^{c}$ be the conatural numbers, i.e. $\mathbb{N} + \{\infty\}$. Consider the endofunctor $\Maybe : \SET \to \SET$ (defined in \cref{exa:nno_initial_alg_maybe}), i.e. the functor given on objects by
\begin{align*}
  \Ob{\Maybe} : \Ob{\SET} &\to \Ob{\SET}
  \\
  X &\mapsto 1 + X.
\end{align*}
The functions
\begin{align*}
zero^{c} : \mathbf{1}&\to \mathbb{N}^{c} \\ 
           \star&\mapsto 0,\\
succ^{c} : \mathbb{N}^{c}&\to \mathbb{N}^{c} \\
          x & \mapsto 
\begin{cases}
n+1,\quad  \text{ if } x := n\in\mathbb{N},\\
\infty,\quad  \text{ if } x := \infty.\\
\end{cases}
\end{align*}
form a $\Maybe$-algebra. However, show that $(\mathbb{N}^{c},zero^{c},succ^{c})$ is not an initial $\Maybe$-algebra.
\end{exer}

\begin{exer}[\textbf{Fusion property}, \cref{sol:fusion-property}]\label{exer:fusion-property}
  Let $F:\CC\to\CC$ be an endofunctor and let $(\Initalg F, \In)$ be an initial algebra. Show that
  for $F$-algebras $(X,\phi)$ and $(Y,\psi)$ and $f\in\CHom{\CC}{X}{Y}$, we have 
\[
\co{\phi}{f} = \co{F(f)}{\psi} \implies \co{\catam{\phi}}{f} = \catam{\psi}.
\]
This is summarized in the following diagram:
\begin{equation}\label{eq:initial-f-alg-composition}
\begin{tikzcd}
F\Initalg{F} 
\arrow[r, "\In"] 
\arrow[d,swap, "F\catam{\phi}"]
\ar[dd, bend right=60, "F\catam{\psi}"']
&
\Initalg{F}
\arrow[d, "\catam{\phi}"]
\ar[dd, bend left=60, "\catam{\psi}"]
\\
FX
\arrow[r, swap, "\phi"]
\ar[d, "Ff"'] 
&
X \ar[d, "f"]
\\
FY \ar[r, "\psi"']
&
Y
\end{tikzcd}
\end{equation}
\end{exer}

\begin{thm}[\textbf{Lambek's theorem}]
  Let $F:\CC\to\CC$ be an endofunctor and let $(\mu^F, \In)$ be an initial algebra. Then, $\In$ is an isomorphism whose inverse is given by $\Inv{\In} = \catam{F(\In)}$.
\end{thm}

\begin{exer} 
  Prove Lambek's theorem. 
\end{exer}

\begin{exer}[\cref{sol:initialalg_for_idfun_with_initialob}]\label{exer:initialalg_for_idfun_with_initialob} Let $\CC$ be a category with an initial object $\bot$. Show that $(\bot, \Id[\bot])$ is the initial algebra for the identity (endo)functor on $\CC$.
\end{exer}

\begin{exer}\label{exer:initialalg_for_list}
  Let $A$ be a set and define $F$ to be the functor induced by 
\begin{align*}
  F_A:\SET &\to\SET
\\
  X &\mapsto 1 + (A\times X).
\end{align*}

\begin{enumerate}
\item \label{enum:list-alg} Show that an $F_A$-algebra consists of a triple $(X,n,c)$, where $X$ is a set, $n\in X$ is an element of $X$, and $c : A \times X \to X$ is a function.
\item Show that the initial $F_A$-algebra is given by the set $\List(A)$ of $A$-valued lists, together with constants $\nil \in \List(A)$ and $\cons : A \times \List(A) \to \List(A)$.
\item Given any other $F_A$-algebra $(X,n,c)$, unpack what it means for the square of \cref{dfn:alg-hom} to commute.
Compare it to a definition of a function $f : \List(A) \to X$ by pattern matching.
\end{enumerate}

\end{exer}

\begin{rem}
  In the case of lists, the operator $\catam{\_}$ is also known as |fold|, which in Haskell is defined as follows:
  \begin{lstlisting}
    fold            :: (a → b → b) → b → ([a] → b)
    fold f v   []   =  v
    fold f v (x:xs) =  f x (fold f v xs)
  \end{lstlisting}
  Compare the input of |fold| to the data of an $F_A$-algebra given in \cref{enum:list-alg} of \cref{exer:initialalg_for_list}.
\end{rem}


\begin{exer}\label{exer:list-functions-as-fold}
  Define the following functions as a catamorphism, that is, using |fold|.
  In each case, draw the diagram corresponding to Diagram~\ref{eq:initial-f-alg} of \cref{dfn:initial-alg}.
  \begin{enumerate}
  \item |sum :: [Int] →  Int|
  \item |product :: [Int] →  Int|
  \item |and :: [Bool ] →  Bool|
  \item |or :: [Bool ] →  Bool|
  \item |(++) :: [a] →  [a] →  [a]|
  \item |length :: [a] →  Int|
  \item |reverse :: [a] →  [a]|
  \item |map :: (a →  b) →  ([a] →  [b])|
  \item |filter :: (a →  Bool ) →  ([a] →  [a])|
  \end{enumerate}
  Solutions are given in \cite[\S2]{DBLP:journals/jfp/Hutton99}.

  Hint: a systematic approach to reformulating functions on lists defined by explicit recursion in terms of |fold| is described in \cite[\S3.3]{DBLP:journals/jfp/Hutton99}.
\end{exer}

\begin{exer}[\cref{sol:list-concat-nil}, see also \cite{DBLP:journals/scp/Malcolm90}]
  \label{exer:list-concat-nil}
  Consider the function |(++) :: [a] →  [a] →  [a]| defined in \cref{exer:list-functions-as-fold},
  and |l :: [a]|.
  \begin{enumerate}
  \item Show that |nil ++ l = l|.
    
  \item Show that |l ++ nil = l|.
  \end{enumerate}
\end{exer}

\begin{exer}[{\cite[\S3.1]{DBLP:journals/jfp/Hutton99}}]
  Show that |(+1) . sum = fold (+) 1|, by showing that |(+1) . sum| makes Diagram~\ref{eq:initial-f-alg} of \cref{dfn:initial-alg} commute.
\end{exer}

% \begin{exer}
%   An exercise about (the limits of) representing functions as catamorphisms (or rewriting functions using `fold`):
%   \begin{enumerate}
%     \item We can represent numbers as big-endian binary numbers, in the set $ \List(\{0, 1\}) $: the lists with elements in the set $ \{0, 1 \} $. For example, $ 13 $ becomes $ [1, 1, 0, 1] $, which we represent as |(cons 1 (cons 1 (cons 0 (cons 1 nil))))|. We define the function $ \texttt{bin2int}: \List(\{0, 1\}) \to \NN $, that converts big-endian binary representations to positive integers. For example, |(bin2int (cons 1 (cons 0 nil))) = 2| and |(bin2int (cons 1 (cons 1 (cons 0 (cons 1 nil))))) = 13|.
%    
%       Show that we can give |bin2int| as a catamorphism. i.e. with $ F $ the functor for which $ \List(\{0, 1\}) $ is an initial algebra, show that there exists an $ F $-algebra $ (\NN, f) $ such that $ \catam f = \texttt{bin2int} $.
%
%     \item We can also represent a number as a little-endian binary number. Then 13 becomes $ [1, 0, 1, 1] $, which we represent as |(cons 1 (cons 0 (cons 1 (cons 1 nil))))|. We define the function $ \texttt{bin2int2}: \List(\{0, 1\}) \to \NN $, that converts little-endian binary representations back to positive integers.
%    
%       Why can't we give |bin2int2| as a catamorphism?
%
%       \textit{(Hint: how would such a catamorphism calculate the values for $ [1] $, $ [0, 1] $ and $ [0, 0, 1] $?)}
%
%     \item Show that there exists an $ F $-algebra $ (\NN \times \NN, f) $ such that $ \projl \circ \catam{f} = \texttt{bin2int2} $, with $ \projl $ the projection on the first coordinate.
%   \end{enumerate}
% \end{exer}

\begin{exer}\label{exer:initialalg_for_btree} Let $A$ be a set and define $F$ to be the functor induced by 
\[
F_A:\SET\to\SET : X\mapsto A + (X\times X).
\]
Show that the initial $F_A$-algebra is given by the set $\BinTree(A)$ of $A$-valued binary trees.
\end{exer}

\begin{rem} Notice that in \cref{exer:initialalg_for_list} and \cref{exer:initialalg_for_btree}, we can consider the functor $F_A$ as a bifunctor where we vary $A$, i.e.
\[
F: \SET\to \SET\to \SET: (A,X)\mapsto F_A(X).
\]
In particular, under the assumption that for every $A\in\SET$ the initial $F_A$-algebra exists, we can wonder if the assignment 
\[
\Ob{\SET} \to \Ob{\SET} : A\mapsto \mu F_A ,
\]
can be extended to a functor. The following exercise answers this question positively for arbitrary categories.
\end{rem}

\begin{exer}[\cref{sol:initialalg_for_bifunctor_functor}]\label{exer:initialalg_for_bifunctor_functor} Let $F:\CC\to\CC\to\CC$ be a bifunctor such that for any object $A\in\CC$, the initial algebra for the functor induced by 
\[
F_A : \CC\to\CC : X\mapsto F(A,X),
\]
exists. Show how
\[
\Ob{\CC} \to \Ob{\CC} : A\mapsto \mu F_A ,
\]
induces a functor.
\end{exer}



\section{Fusion Property}\label{sec:fusion}

The fusion property of \cref{exer:fusion-property} can be used to ``fuse'' a composition of functions into one function, possibly leading to more efficient code.

We are going to exemplify this using the datatypes of lists and of natural numbers.
Recall that $(\NN, [\Zero,\Succ])$ is the initial $\Maybe$-algebra.
We also write $(+1)$ for $\Succ$.

\begin{reading*}
  The content of this section is very much inspired by \cite[\S3.2]{DBLP:journals/jfp/Hutton99}.
  You are strongly encouraged to read that section before reading the present section.

  The fusion property is called ``promotion theorem'' in Malcolm's work~\cite{DBLP:journals/scp/Malcolm90}.
\end{reading*}

\begin{exer}[See also \cite{DBLP:journals/scp/Malcolm90}]
  Consider the function |(++) :: [a] →  [a] →  [a]| defined in \cref{exer:list-functions-as-fold}.
  Show that |(l ++ m) ++ n = l ++ (m ++ n)| for any |l, m, n :: [a]|.
\end{exer}


\begin{exer}[{\cite[\S3.2]{DBLP:journals/jfp/Hutton99}}]
  Show that |(+1) . sum = fold (+) 1|, by using the fusion property of \cref{exer:fusion-property}.
\end{exer}

\begin{solution}
  We also write $\sum$ for |sum|. Note that |sum| is defined as the catamorphism $\catam{[\Zero, (+)]}$.
  The situation is summarized in the following diagram
  
  \[
    \begin{tikzcd}[row sep = huge, column sep = huge]
      \{*\} + \NN \times \List(\NN)
      \ar[d, "\Id + \Id \times \sum" description]
      \ar[r, "{[\nil, \cons]}"]
      &
      \List(\NN)
      \ar[d, "\sum" description]
      \ar[dd, bend left, "\catam{[1, (+)]}"]
      \\
      \{*\} + \NN \times \NN
      \ar[r, "{[\Zero, (+)]}"]
      \ar[d, "{\Id + \Id \times (+1)}" description]
      &
      \NN
      \ar[d, "{(+1)}" description]
      \\
      \{*\} + \NN \times \NN
      \ar[r, "{[1, (+)]}"]
      &
      \NN 
    \end{tikzcd}
  \]

  In order to show that |(+1) . sum = fold (+) 1|, in the diagram expressed as $\co{\sum}{(+1)} = \catam{[1, (+)]}$, we need to show that the lower rectangle commutes---this is what  \cref{exer:fusion-property} says.

  There are two cases to consider: the case of $*$, and the case of a pair $(m,n) \in \NN\times\NN$.
  In the case of $*$, we obtain
  \[
    (+1)~0 = 1
  \]
  and in the case of $(m,n)$, we obtain
  \[
    (+1)~ ((+)~ m~ n) = (+)~ m~ ((+1)~n)
  \]
  both of which hold.
  
\end{solution}

\begin{exer}
  Prove that |map g . map f = map (g . f)| using the fusion property.
\end{exer}


\begin{reading*}
  In \cref{sec:initial-algs,sec:fusion}, we have only looked at functions defined via \emph{iteration}, that is, functions defined as \emph{catamorphisms} of some $F$-algebra.
  While \emph{primitive recursive} functions can always be expressed as a catamorphism via tupling (see, e.g., \cite[\S4]{DBLP:journals/jfp/Hutton99} and \cite[\S3.1]{vene_phd}), it is more natural specify them via a slightly more sophisticated universal property, explained in detail in Vene's dissertation~\cite[Chapter~3]{vene_phd}.

  The interested reader might also study the tutorial by Fokkinga~\cite{Fokkinga_homo-cata} or the paper by Meijer et al.~\cite{DBLP:conf/fpca/MeijerFP91}.
  The paper \cite{DBLP:journals/scp/Malcolm90} by Malcolm contains more examples.
  
  A guide to further literature on recursion operators is given in \cite[\S6]{DBLP:journals/jfp/Hutton99}.
\end{reading*}

\chapter{Terminal Coalgebras and Coinductive Datatypes}\label{sec:coinductive}


\begin{reading*}
  We only give a brief introduction to (terminal) coalgebras in this section.
  A more systematic exploration of the topic is given in~\cite{DBLP:conf/fpca/MeijerFP91}.

  The paper \cite{DBLP:journals/scp/Malcolm90} by Malcolm contains further examples.
\end{reading*}


In this section we introduce (terminal) coalgebras which allows us to define coinductive data types.

\begin{dfn} Let $F:\CC\to \CC$ be an endofunctor. An \textbf{$F$-coalgebra} consists of the following data:
\begin{itemize}
\item An object $X\in \CC$.
\item A morphism $\phi \in \CHom{\CC}{X}{F(X)}$.
\end{itemize}
\end{dfn}
Notice that an $F$-algebra consists of a morphism $F(X)\to X$, while an $F$-coalgebra consists of a morphism $X\to F(X)$ in the other direction.

\begin{dfn} Let $F:\CC\to \CC$ be an endofunctor and $(X,\phi)$ and $(Y,\psi)$ be $F$-coalgebras. A \textbf{($F$-coalgebra) homomorphism} from $(X,\phi)$ to $(Y,\psi)$ is a morphism $f\in \CHom{\CC}{X}{Y}$ such that the following diagram commutes:
\[
\begin{tikzcd}
X
\arrow[r, "\phi"] 
\arrow[d,swap, "f"]
& FX
\arrow[d, "F(f)"] 
\\
Y
\arrow[r, swap, "\psi"] 
& FY
\end{tikzcd}
\]
\end{dfn}

\begin{exer} Define the category $\COALG{F}$ of $F$-coalgebras analogously to the category $\ALG{F}$ of $F$-algebras (as in \cref{definition:category_of_Falgebras}).
\end{exer}

\begin{dfn} Let $F$ be an endofunctor on $\CC$. The \textbf{terminal $F$-coalgebra} is the terminal object in $\COALG{F}$ which we denote by $(\Terminalcoalg{F}, \Out)$.
  
For $(X,\phi)$ an arbitrary $F$-coalgebra, we denote the unique morphism $(X,\phi) \to (\Terminalcoalg{F}, \Out)$ by $\anam{\phi}$, and we call a morphism of this form an \textit{anamorphism} (instead of a catamorphism as in \cref{dfn:initial-alg}).
\end{dfn}

\begin{exer} Spell out what it means for a coalgebra to be the terminal coalgebra.
\end{exer}

\begin{exer} Let $F:\CC\to\CC$ be an endofunctor and let $(\Terminalcoalg{F}, \Out)$ be a terminal coalgebra. Show that the following properties holds:
\begin{enumerate}
\item $\Id = \anam{\Out}$.
\item For $F$-algebras $(X,\phi)$ and $(Y,\psi)$ and $f\in\CHom{\CC}{X}{Y}$, we have 
\[
  \co f \psi   = \co \phi {F(f)} \implies \co{f}{\anam{\psi}} = \anam{\phi}.
\]
This is summarized in the following diagram:
\[
\begin{tikzcd}
X
\arrow[r, "\phi"] 
\arrow[d,swap, "f"]
\ar[dd, "\anam{\phi}"', bend right = 60]
&
FX
\arrow[d, "Ff"]
\ar[dd, "F\anam{\phi}", bend left = 60]
\\
Y
\arrow[r, swap, "\psi"]
\ar[d, "\anam{\psi}"']
&
FY
\ar[d, "F\anam{\psi}"]
\\
\Terminalcoalg{F}
\ar[r, "\Out"]
&
F\Terminalcoalg{F}
\end{tikzcd}
\]
\end{enumerate} 
\end{exer}

\begin{thm}[\textbf{Dual of Lambek's theorem}] Let $F:\CC\to\CC$ be an endofunctor and let $(\Terminalcoalg{F}, \Out)$ be a terminal coalgebra. Then $\Out$ is an isomorphism whose inverse is given by $\Inv{\Out} = \anam{F(\Out)}$.
\end{thm}

\begin{exer}
  Prove the dual of Lambek's theorem. 
\end{exer}

\begin{exer}\label{exer:terminalalg_for_idfun_with_terminalob} Let $\CC$ be a category with a terminal object $\top$. Show that $(\top, \Id[\top])$ is a terminal coalgebra for the identity (endo)functor on $\CC$.
\end{exer}

\begin{exer}[\cref{sol:conatural_numbers_terminal_coalgebra}] \label{exer:conatural_numbers_terminal_coalgebra} Let $F$ be the functor induced by 
\[
F:\SET\to\SET : X\mapsto 1 + X.
\]
Show that the terminal $F$-coalgebra is given by the following data:
\begin{itemize}
\item The underlying object is given by the set $\mathbb{N} + \{\infty\}$ of natural numbers with infinity.
\item The underlying function is given by the predecessor defined as follows: 
\begin{align*}
  \mathbb{N} + \{\infty\} &\to 1 + \mathbb{N} + \{\infty\}
  \\
  0 &\mapsto \star
  \\
  s(n) &\mapsto n
  \\
  \infty &\mapsto \infty
\end{align*}
where $\star$ is the unique element of $1$.
\end{itemize}
\end{exer}


\begin{exa}[Streams]
  The codata type of streams over a given set $A$ is given by the terminal coalgebra $(\Terminalcoalg{F_A}, \Out)$ of the functor $F_A (X) := A \times X$.
  We write $\Stream(A)$ for $\Terminalcoalg{F_A}$.
  The functions $\head : \Stream(A) \to A$ and $\tail : \Stream(A) \to \Stream(A)$
  are given by
  \begin{align*}
    \head &= \co{\Out}{\projl} : \Stream(A) \to A
    \\
    \tail &= \co{\Out}{\projr} : \Stream(A) \to \Stream(A)
  \end{align*}
  respectively. Put differently (recall the definition of the product in \cref{def:binproduct}), we have
  \[
    \Out = \langle \head, \tail \rangle : \Stream(A) \to A \times \Stream(A)
  \] 
  Given any two functions
  \begin{align*}
    h : C \to  A
    \\
    t : C \to C
  \end{align*}
  the anamorphism $\anam{\langle h, t \rangle }$ is the unique solution
  $f : C \to  \Stream(A)$
  of the equation system
  \begin{align*}
    \co{f} \head &= h
    \\
    \co f \tail &=  \co t f
  \end{align*}
  that is, the unique function $f : C \to \Stream(A)$ making the following square commute:
  \[
    \begin{tikzcd}[column sep = large]
      C
      \ar[r, "{\langle h, t \rangle}"]
      \ar[d, "\anam{\langle h, t \rangle}"']
      &
      A \times C
      \ar[d, "\Id \times \anam{\langle h, t \rangle}"]
      \\
      \Stream(A)
      \ar[r, "{\langle \head, \tail \rangle}"]
      &
      A \times \Stream(A).
    \end{tikzcd}
  \]
  
\end{exa}

\begin{exer}[\cref{sol:stream-of-nats}]\label{exer:stream-of-nats}
  Define, as an anamorphism, the function $\nats : \NN \to \Stream(\NN)$ which returns the stream of all natural
  numbers starting with the natural number given as the argument.

  What do you need to change to obtain instead the stream listing \emph{every other} number starting with the given one?
\end{exer}


\begin{exer}[\cref{sol:zip}]\label{exer:zip}
  Define, as an anamorphism, the function $\zip : \Stream(A) \times \Stream(B) \to \Stream(A \times B)$ that zips the argument streams together.
\end{exer}

\begin{exer}
  Define, as an anamorphism, the function $\map : (A \to B) \to \Stream(A) \to \Stream(B)$.
\end{exer}

\begin{exer}
  For a fixed set $A$, consider the functor given on objects by $F(X) := 1 + A \times X$.
  Its terminal $F$-coalgebra is the datatype $\Colist(A)$ of potentially infinite lists over elements in $A$, with suitable destructors $\head$ and $\tail$ and an ``exception'' in case the colist is empty.

  Define a function $\Colist(A) \to  \mathbb{N} + \{\infty\}$ counting the elements in a colist.  
\end{exer}

\begin{exer}
  Try to define $\filter : (A \to \Bool) \to \Stream(A) \to \Stream(A)$.
  What goes wrong?

  One might think that $\filter : (A \to \Bool) \to \Stream(A) \to \Colist(A)$ works instead.
  Does it?

  What about $\filter : (A \to \Bool) \to \Stream(A) \to \Stream(A+1)$?
  
\end{exer}

%%% Local Variables:
%%% mode: latex
%%% TeX-master: "CT4P"
%%% End:
