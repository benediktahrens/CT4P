\chapter{Natural Transformations}
\label{sec:nat-trans}

Natural transformations are often considered the heart of category theory, and for good reason 
— the very subject was originally developed to express such transformations.
They provide the language to express many of the field's fundamental constructions.
Moreover, in the context of programming, natural transformations are crucial: they serve as the mathematical model for parametric polymorphism when a category is used to represent a programming language.

\section{Definition and First Examples}

\begin{dfn} Let $F,G: \CC\to\DD$ be functors. A \textbf{natural transformation} $\alpha$ from $F$ to $G$ consists of the following data:
\begin{itemize}
\item For each $X\in \Ob{\CC}$, a morphism $\alpha_X \in \CHom{\DD}{F(X)}{G(X)}$.
\end{itemize}
Moreover, this data should satisfy the following \textit{naturality condition}:\\
For each $f\in \CHom{\CC}{X}{Y}$, the following diagram should commute:
\begin{center}
\begin{tikzcd}
F(X) \arrow[r, "\alpha_X"] \arrow[d,swap, "F(f)"] & G(X) \arrow[d, "G(f)"]\\
F(Y) \arrow[r,swap, "\alpha_Y"] & G(Y)
\end{tikzcd}
\end{center}
\end{dfn}

\begin{nota} If $F,G:\CC\to\DD$ are functors, a natural transformation $\alpha$ from $F$ to $G$ is denoted as $\NatTrans{\alpha}{F}{G}$ or 
\begin{center}
\begin{tikzcd}[column sep=huge]
\CC
  \arrow[bend left=50]{r}[name=U,label=above:$\scriptstyle F$]{}
  \arrow[bend right=50]{r}[name=D,label=below:$\scriptstyle G$]{} &
\DD
  \arrow[shorten <=5pt, Rightarrow,to path={(U) -- node[label=right:$\alpha$] {} (D)}]{}
\end{tikzcd}
\end{center}
\end{nota}

\begin{exer}
Construct a natural transformation from the $\Maybe$ functor to the $\List$ functor, and conversely.
\end{exer}

\begin{exa} (\textbf{Currying}) Let $X$ be a set. Let $F := \SET(X, -)\times X : \SET\to\SET$ be the functor induced by the following data (on objects):
\[
Y\mapsto \SET(X,Y)\times X.
\]
The evaluation defines a natural transformation $\NatTrans{ev}{F}{\Id[\SET]}$ as follows:
\[
ev_Y : \SET(X,Y) \times X \to Y : (f,x) \mapsto f(x).
\]
Show that this indeed satisfies the naturality condition.
\end{exa}

\begin{exer}
  Given two preordered sets $(X,\leq_X)$ and $(Y, \leq_Y)$ and two functors $f, g : \PREtoCAT(X,\leq_X) \to \PREtoCAT(Y,\leq_Y)$ between their associated categories, what is a natural transformation from $f$ to $g$?
\end{exer}

Recall that for every object $d : \DD$ in a category gives rise to the constant functor (see \cref{exer:const-functor}).

\begin{exer}
  Can you define a natural transformation that counts the occurrences of elements from $X$ in a list?
  That is, for any set $X$, we would like to have a function of type $\List(X) \to X \to \NN$.
\end{exer}

\begin{exer}
Can you define a natural transformation that counts the number of elements in a list?
That is, for any set $X$, construct a function $len_X : \List(X) \to \NN$ and show that this is natural in $X$.
\end{exer}

\begin{exer}
  Let $(M,m,e)$ be a monoid and let $\MONtoCAT(M,m,e)$ be its corresponding category. Recall from \cref{ex:monoid_functors} that a functor from $\MONtoCAT(M,m,e)$ to $\SET$ is a set $X$ together with an action of $M$ on $X$, i.e. a function $\mu: M\times X\to X$ such that 
  \[
    \forall x\in X: \mu(e,x) = x, \quad \forall n_1,n_2\in M, x\in X: \mu(n_1, \mu(n_2,x)) = \mu(m(n_1,n_2), x).
  \]
  We will call a set $ X $ with an action of $ M $ on $ X $ an $ M $-set.
  Describe the natural transformations between $M$-sets.
\end{exer}

\begin{exer}
  Let $(X,\leq_X)$ and $(Y,\leq_Y)$ be posets.
  Recall from \cref{ex:poset_functors} that a functor between posets corresponds with an order-preserving function, i.e. $x_1 \leq_X x_2 \implies f(x_1) \leq_Y f(x_2)$.
  Describe the natural transformations between order-preserving functions.
\end{exer}

\begin{exer}
Let $\CC$ be a category and $\mathsf{bool}$ be the category with $2$ distinct objects $a,b$ and no non-trivial morphisms.
For every two objects $x, y \in \CC$, there is a functor $J : \mathsf{bool} \to \CC$ with $J(a) := x$ and $J(b) := y$.

Assume $x + y$ is a coproduct of $x$ and $y$.
Construct a natural transformation $\alpha$ from $J$ to the constant functor $\mathsf{const}_{(x + y)} : \mathsf{Bool} \to \CC$.
Furthermore, can you describe the universal property of the coproduct in terms of $\alpha$?
\end{exer}

\section{Functor Categories}

\begin{dfn}\label{dfn:nattrans_id} Let $F:\CC\to\DD$ be a functor. The \textbf{identity natural transformation} $\Id[F]$ on $F$ is given by the following data:
\[
\forall X\in\Ob{\CC}: (\Id[F])_{X} := \Id[F(x)].
\]
\end{dfn}

\begin{exer} Show that for any functor $F:\CC\to\DD$, the identity natural transformation $\Id[F]$ satisfies the properties of a natural transformation.
\end{exer}

\begin{dfn}\label{dfn:nattrans_comp} Let $F,G,H: \CC\to\DD$ be functors and $\NatTrans{\alpha}{F}{G}$, $\NatTrans{\beta}{G}{H}$ be natural transformations. The \textbf{(vertical) composition} of $\alpha$ and $\beta$ is the natural transformation $\Comp{\alpha}{\beta}$ is given by the following data:
\[
\forall X\in\Ob{\CC}: (\co{\alpha}{\beta})_{X} := \co{\alpha_X}{\beta_X}.
\]
\end{dfn}

\begin{exer} Show that for any functors $F,G,H: \CC\to\DD$ and $\NatTrans{\alpha}{F}{G}$, $\NatTrans{\beta}{G}{H}$ natural transformations, the (vertical) composition of $\alpha$ and $\beta$ satisfies the properties of a natural transformation.
\end{exer}

\begin{dfn} Let $\CC,\DD$ be categories. The \textbf{category of functors} or the \textbf{functor category} from $\CC\to\DD$, denoted by $Fun(\CC,\DD)$ or $[\CC,\DD]$, is given by the following data:
\begin{itemize}
\item An object is a functor $F:\CC\to\DD$.
\item Given functors $F, G:\CC\to\DD$, a morphism from $F$ to $G$ is a natural transformation $\NatTrans{\alpha}{F}{G}$.
\item The identity morphism on $F$ is given by the identity natural transformation $\Id[F]$ defined in \cref{dfn:nattrans_id}.
\item The composition of $\alpha$ and $\beta$ is given by the composition $\co{\alpha}{\beta}$ defined in \cref{dfn:nattrans_comp}.
\end{itemize}
\end{dfn}

\begin{exer} Show that for any two categories $\CC$ and $\DD$, the functor category from $\CC$ to $\DD$ satisfies the properties of a category.
\end{exer}

\begin{exer}
Let $\CC$ be the category generated by two objects $(V, E)$ and with $2$ morphisms $s, t : E \to V$.
Can you describe the category $[\CC, \SET]$?
\end{exer}

\begin{dfn}
A natural transformation $\alpha : F \Rightarrow G$ is a \textbf{natural isomorphism} if for each $X\in\Ob{\CC}$, we have that $\alpha_X$ is an isomorphism in $\DD$.
\end{dfn}

\begin{exer}
Let $F, G : \CC \to \DD$ be functors.
Show that a natural isomorphism from $F$ to $G$ is equivalent to an isomorphism between $F$ and $G$ in the functor category $[\CC, \DD]$.
\end{exer}

\begin{dfn}\label{dfn:nattrans_horcomp} Let $F,G : \CC\to\DD$ and $\tilde{F},\tilde{G}:\DD\to\EE$ be functors and $\NatTrans{\alpha}{F}{G}, \NatTrans{\beta}{\tilde{F}}{\tilde{G}}$ be natural transformations. The \textbf{horizontal composition} (also called the \textbf{Godement product}) of $\alpha$ and $\beta$, denoted by $\beta \bullet \alpha$, is defined as:
\begin{equation}\label{eqn:nattrans_horcomp}
\forall X\in \Ob{\CC}: (\beta\bullet\alpha)_X := \co{\tilde{F}(\alpha_X)}{\beta_{G(X)}}.
\end{equation}
\end{dfn}

\begin{exer} Show that $\alpha\bullet\beta$ (defined as in \cref{dfn:nattrans_horcomp}), is indeed a natural transformation.
\end{exer}

\begin{exer} Show the following property: 
\[
\forall X\in \Ob{\CC}: (\beta\bullet\alpha)_X = \co{\beta_{F(X)}}{\tilde{G}(\alpha_X)}.
\]
Hint: Write the equality as a (not-known commutative) square.
\end{exer}

\section{Equivalence of Categories}
Recall that objects $X,Y\in\Ob{\CC}$ are isomorphic if there exist morphisms $f\in\CHom{C}{X}{Y}$ and $g\in\CHom{C}{Y}{X}$ such that $\co{f}{g} = \Id[X]$ and $\co{g}{f} = \Id[Y]$. So in particular we have the notion of an isomorphism in the category $\CAT$ of categories. Spelled out, this means categories $\CC$ and $\DD$ are isomorphic if there exist functors $F:\CC\to\DD$ and $G:\CC\to\DD$ such that $\co{F}{G}= \Id[\CC]$ and $\co{G}{F} = \Id[\DD]$.\\
However, the following exercise shows that isomorphism of categories is not the correct notion of \textit{equivalence/sameness} between categories:\\
Let $\FINSET$ be the category whose objects are given by finite sets and whose morphisms are given by functions\footnote{Notice that the objects of $\FINSET$ form a subset of the objects of $\SET$, but given any two finite sets $X,Y \in \Ob{\FINSET}$, we have $\CHom{\FINSET}{X}{Y}$ = $\CHom{\SET}{X}{Y}$. We say in this case that $\FINSET$ is a (full) subcategory of $\SET$.}. That this is a category follows since $\SET$ is a category.\\
Let $\Catb{FinOrd}$ be the category whose objects are given by sets of the form 
\[
[n] := \left\{0,1,\cdots,n-1\right\},
\]
and whose morphisms are given by functions between these sets.\\
Every finite set $X$ is always in bijection with a set of the form $[n]$ (where $n$ is the size $\vert X\vert$ of $X$). For each set $X$, we fix a bijection $\phi^X: X\to [\vert X\vert]$. Consequently, we have a functor:
\begin{dfn} Let $U: \FINSET\to \Catb{FinOrd}$ be the functor specified by the following data:
\begin{itemize}
\item For $X\in \Ob{\FINSET}$, we define $U(X) := [\vert X\vert]$.
\item For $f\in \CHom{\FINSET}{X}{Y}$, we define $U(f): [\vert X\vert]\to [\vert Y\vert]$ as the unique function such that the following diagram commutes:
\begin{center}
\begin{tikzcd}
X \arrow[r, "\phi^X"] \arrow[d,swap, "f"] & {[\vert X\vert]} \arrow[d, "U(f)"] \\
Y \arrow[r,swap, "\phi^Y"] & {[\vert Y\vert]}
\end{tikzcd}
\end{center}
\end{itemize}
\end{dfn}

\begin{exer} Show that $U: \FINSET\to \Catb{FinOrd}$ is indeed a functor. In particular, you have to show that $U$ is well-defined on the morphisms.
\end{exer}

In order to show that $U$ is not an isomorphism, one can use the following lemma/exercise:
\begin{exer} Show that a functor $F:\CC\to\DD$ is an isomorphism if and only if $F$ satisfies the following properties:
\begin{itemize}
\item $F$ is injective on objects, i.e. 
\[
\forall X,Y\in\Ob{\CC}: F(X) = F(Y) \implies X=Y.
\]
\item $F$ is surjective on objects, i.e. 
\[
\forall Y\in\Ob{\DD}: \exists X\in\Ob{\CC} : F(X) = Y.
\]
\item $F$ is faithful, i.e. the following functions are injective
\[
\forall X,Y\in\Ob{\CC}: \CHom{\CC}{X}{Y} \xrightarrow{F_{X,Y}} \CHom{\DD}{F(X)}{F(Y)} : f\mapsto F(f)
\]
\item $F$ is full, i.e. for all $X,Y\in \Ob{\CC}$, $F_{X,Y}$ is surjective.
\end{itemize}
\end{exer}

\begin{exer} Show that $U$ is not an isomorphism, i.e. state which part of an isomorphism fails and give a concrete example that it fails.
\end{exer}

\begin{rem} The problem with $U$ (in the sense that it is not an isomorphism) is that multiple (finite) sets are mapped to the same set. For this reason, a good notion of equivalence between categories should not be injective on objects. Also, which is not clear from this example, we should also weaken the condition of $F$ being surjective on objects. Instead, we need that $F$ is \textbf{essentially surjective on objects}:
\[
\forall Y\in\Ob{\DD}: \exists X\in\Ob{\CC} : F(X) \cong Y.
\]
\end{rem}
Motivated by the remark, we define:
\begin{dfn} Categories $\CC$ and $\DD$ are \textbf{equivalent} if there exists a pair of functors $(F:\CC\to\DD, G:\DD\to\CC)$ such that there exists natural isomorphisms 
\[
\co{F}{G} \to \Id[\CC], \quad \co{G}{F} \to \Id[\DD]
\]
\end{dfn}

So, although $U$ is not an isomorphism, it does induce an equivalence of categories:
\begin{exer} Show that $U$ induces an equivalence of categories.
\end{exer}

\begin{exer} Let $\CC$ be the category whose objects are categories with a unique object and whose morphisms are functors between these one-object categories, i.e. $\CC$ is the (full) subcategory of $\CAT$ generated by the categories with a unique object. Show that $\CC$ is equivalent to the category $\MON$ of monoids.
\end{exer}
\begin{exer}
In the previous exercise, what happens if we do not consider $\CC$ to consist of those categories with a unique object, but with a unique object up to isomorphism? In other words, consider $\CC$ as the category whose objects are categories $\DD$ which satisfy the following property: 
\[
\forall X,Y \in \Ob{\DD}: X\cong Y.
\]
\end{exer}

The following exercise gives a characterization of a functor being an equivalence. However, in order to show this, one has to use the axiom of choice which means (informally) that if the following property holds:
\[ 
\exists x: P(x),
\]
then we can fix some $x$ such that $P(x)$ holds.
\begin{exer} Show that a functor $F:\CC\to\DD$ induces an equivalence of categories if and only if it is essentially surjective on objects and fully faithful.
\end{exer}

\begin{exer}
  Show that the category $\MAT$ is equivalent to the category of finite-dimensional real vector spaces and linear maps.
\end{exer}

\section{Representable Functors}

Representable functors are functors into $\SET$ which are isomorphic to the so-called hom-functors \cref{exer:covariant-hom,exer:contravariant-hom}.
\cref{exer:coproducts-as-representable} illustrates that universal properties can be described in terms of representables.
\cref{exer:forgetful-functor-is-representable} illustrates that free constructions can be described in terms of representables.
\cref{exer:representables-on-posets} illustrates that objects in a category are completely determined by their representable functors.

\begin{exer}
\label{exer:covariant-hom}
Let $\CC$ be a category and $x \in \CC$.
Consider the function $\CC(x,-) : \Ob{\CC} \to \Ob{\SET}$ sending an object $y \in \CC$ to the hom-set $\CC(x,y)$.
For $y, z \in \CC$ and $g \in \CC(y, z)$ define
\[
\CC(x, g) : \CC(x,y) \to \CC(x,z) : f \mapsto g \circ f.
\]
Show that this defines a functor $\CC(x,-) : \CC \to \SET$.
This functor is called the \textbf{covariant hom-functor}.
\end{exer}

Recall that $\op{\CC}(x,y) := \CC(y,x)$.

\begin{exer}
\label{exer:contravariant-hom}
Show that for every $y : \CC$, the function $\CC(-,y) : \Ob{\CC} \to \Ob{\SET}$ sending an object $x \in \CC$ to the hom-set $\CC(x,y)$ induces a functor of type $\CC(-,y) : \op{\CC} \to \SET$.
This functor is called the \textbf{contravariant hom-functor}.
\end{exer}

\begin{exer}
Show that every category $\CC$ induces a functor
\[
\CC(-,-) : \op{\CC} \times \CC \to \SET.
\]
\end{exer}

\begin{exer}
Let $F : \CC \to \DD$ be a functor.
Can you construct a natural transformation from $\CC(-,-)$ to $\DD(-,-) \circ ({\op{F} \times F})$?
\end{exer}

\begin{dfn}
A functor $F : \op{\CC} \to \SET$ is \textbf{representable} if there exists an object $c \in \CC$ and a natural isomorphism $\theta : \CC(-,c) \simeq F$.
We say that $(c, \theta)$ is a \textbf{representation} for $F$.
\end{dfn}

\begin{lemma}
A representation for a functor $F$ into $\SET$ is unique up to isomorphism.
\end{lemma}

\begin{exer}
\label{exer:coproducts-as-representable}
Let $\CC$ be a category and $x,y \in \CC$ objects.
Show that the following are equivalent:
\begin{enumerate}
\item the coproduct of $x$ and $y$ exists;
\item the functor $\CC(-,x) \times \CC(-,y)$ is representable.
\end{enumerate}
\end{exer}

Recall that $\MON$ denotes the category of monoids and monoid homomorphisms.
Observe that for every set $A$, $\List(A)$ has a monoid structure whose multiplication is given by the concatenation of lists and whose unit is the empty/nil list.
See \cref{ch:forgetful-free} for more details.
\begin{exer}[\cref{sol:forgetful-functor-is-representable}]\label{exer:forgetful-functor-is-representable}
Show that the forgetful functor $U : \MON \to \SET$ is representable.
Furthermore, show that for every set $A$ the functor 
\[
\SET(A, U-) : \MON \to \SET,
\]
is representable.
\end{exer}

\begin{exer}[\cref{sol:representables-on-posets}]
\label{exer:representables-on-posets}
Let $L := (L,\leq)$ be a poset and $\mathcal{L} := \POS(L,\leq)$ be its associated category.
Describe the representable functors on $\mathcal{L}$.
\end{exer}