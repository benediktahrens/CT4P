\chapter{Special Morphisms in a Category}

\section{Isomorphisms}
\label{sec:isos}

\begin{reading*}
  In this section, we study properties of arrows in a category.
  More information on this topic is given in \cite[\S 2.7]{barr-wells}.

  Also, \cite[\S 2.2]{pierce} briefly discusses isomorphisms.
\end{reading*}


\begin{dfn}[Isomorphism]
  Given a category $\CC$, objects $a,b \in \Ob{\CC}$ and a morphism $f : a \to b$ in $\CC$, we say that $f$ is an \textbf{isomorphism} when there is a morphism $g : b \to a$ (in the other direction!) such that $f \Comp g = \Id$ and $g \Comp f = \Id$.
  We write $f : a \cong b$ for a morphism $f$ that is an isomorphism.

  In this case, we call $g$ the \textbf{inverse} of $f$ and $f$ the inverse of $g$. (The latter is justified by \cref{exer:inverse-iso}.)
\end{dfn}

\begin{exer}[\cref{sol:inverse-iso}]\label{exer:inverse-iso}
  Show that if $f : a \to b$ is an isomorphism with inverse $g : b \to a$, then $g$ is an isomorphism with inverse $f$.
\end{exer}

\begin{exer}[\cref{sol:inverse_uniqueness}]\label{exer:inverse_uniqueness}
  Show that a morphism $f : a \to b$ in $\CC$ is an isomorphism \textbf{in at most one way}, that is, show that its inverse is unique if it exists.
\end{exer}

\begin{exer}[\cref{sol:compofiso}]\label{exer:compofiso}
  Show that the composition of two isomorphisms is an isomorphism.
\end{exer}
% \begin{rem} 
% Since any identity morphism is an isomorphism (check this!), we conclude by \cref{exer:compofiso} that given any category $\CC$, we always get a new category $isos(\CC)$ by restricting the morphisms to be isomorphisms, i.e. \[
% \Ob{isos(\CC)} = \Ob{\CC}, \quad \CHom{isos(\CC)}{X}{Y} = \left\{f \in \CHom{\CC}{X}{Y} \mid f \text{ is an isomorphism}\right\}
% \] 
% and where the identity and composition is the same as in $\CC$.
% \end{rem}


\begin{exer}[\cref{sol:iso-bool}]\label{exer:iso-bool}
  Consider the datatype
\begin{lstlisting}
data BW = Black | White
\end{lstlisting}
Construct two (different!) isomorphisms between |BW| and the type |Bool| of booleans.
\end{exer}

\begin{exer}[\cref{sol:iso_in_sets}]\label{exer:iso_in_sets}
  Describe the isomorphisms in $\SET$.
\end{exer}

\begin{exer}[\cref{sol:iso_in_pos}]\label{exer:iso_in_pos}
  Describe the isomorphisms in $\POS$.
\end{exer}

\begin{exer}[\cref{sol:iso_in_posetcategory}]\label{exer:iso_in_posetcategory}
  Let $(X,\leq)$ be a poset.
  Describe the isomorphisms in $\POS(X,\leq)$.
\end{exer}

\begin{exer}
  Describe the isomorphisms in $\MON$.
\end{exer}

\begin{exer} Let $\mathcal{G}$ be the category generated by the following graph:
\[
\begin{tikzcd}
& w & \\
x & & z \arrow[lu, bend left] \arrow[lu,bend right] \\
& y \arrow[lu] \arrow[ru] &
\end{tikzcd}
\]
Show that the only isomorphisms in $\mathcal{G}$ are the identity morphisms (i.e. the identity paths).
\end{exer}

\begin{exer}[\cref{sol:iso_in_posetcategory}] \label{exer:iso_in_cats_of_nats}
  Describe the isomorphisms in $\POS(\NN,\leq)$, $\SKELFINSET$ and $\MAT$.
\end{exer}

\section{Sections and Retractions}
\label{sec:sections}


\begin{dfn}[Section, Retraction]
  A pair $(s,r)$ of morphisms $s : a \to b$ and $r : b \to a$ in $\CC$ is called a \textbf{section-retraction pair} if $\co{s}{r} = \Id[a]$.

  In such a case, we call $s$ a section and $r$ a retraction.
\end{dfn}

\begin{rem}
  Note that a morphism can be a retraction in more than one way, that is, there can be more than one section $s$ such that $\co{s}{r} = \Id$.
\end{rem}

Intuitively, a section-retraction pair $(s,r)$ of morphisms $s : a \to b$ and $r : b \to a$ in a category $\CC$ provides a way for $a$ to ``live inside'' $b$.
Note that for a given $a$ and $b$ there can be many ways for $a$ to live inside $b$.

\begin{exer}[\cref{sol:section-retraction-bool-int}]\label{exer:section-retraction-bool-int}
  Construct two different section-retraction pairs between the type |Bool| of booleans and the type |Int| of integers (e.g., in Haskell).
\end{exer}



\begin{exer}
 Show that the type |Maybe a| is a retract of the type |[a]|. 
 
 Hint: The idea is that |Nothing| corresponds to the empty list |[]| and that |Just x| corresponds to the one-element list |[x]|. Make this idea precise by writing back and forth functions between these types so that they exhibit |Maybe a| as a retract of |[a]|. 
\end{exer}


\section{Monomorphisms and Epimorphisms}
\label{sec:mono-epi}

\begin{reading*}
See also \cite[p. 134]{leinster} and \cite[\S\S 2.8--2.9]{barr-wells}.
Also, \cite[\S 2.2]{pierce} briefly discusses monomorphisms and epimorphisms.
\end{reading*}

From undergraduate mathematics courses you know what injective and surjective functions between sets are.
The definitions of ``injective'' and ``surjective'' do not carry over to any category (though they do for categories that are, in some sense, ``similar'' to the category of sets).
In this section, we study two properties of morphisms in a category that, in the category of sets, are equivalent to ``injective'' and ``surjective'', respectively.



\begin{dfn}[Monomorphism]
  Let $f : a \to b$ be a morphism in $\CC$. We say that $f$ is a \textbf{monomorphism} if, for any two morphisms $g_1, g_2 : z \to a$, like in the following diagram,
  \begin{center}
    \begin{tikzcd}
    z \arrow[r, "g_2"', shift right] \arrow[r, "g_1", shift left] & a \arrow[r, "f"] & b
    \end{tikzcd}
  \end{center}
  we have
  \[ \co{g_1}{f} = \co{g_2}{f} \text{ implies } g_1 = g_2 .\]
\end{dfn}

\begin{exer}[\cref{sol:mono-inj}]\label{ex:mono-inj}
  In the category of sets, show that a morphism $f : X \to Y$ is a monomorphism if and only if it is injective.
\end{exer}

\begin{dfn}[Epi]
  Let $f : a \to b$ be a morphism in $\CC$. We say that $f$ is an \textbf{epimorphism} if, for any two morphisms $g_1, g_2 : b \to z$, like in the following diagram,
  \begin{center}
    \begin{tikzcd}
    a \arrow[r, "f"] & b \arrow[r, "g_2"', shift right] \arrow[r, "g_1", shift left] & z
    \end{tikzcd}
  \end{center}
  we have
  \[ \co{f}{g_1} = \co{f}{g_2} \text{ implies } g_1 = g_2 .\]
\end{dfn}

\begin{exer}\label{ex:epi-surj}
  In the category of sets, show that a morphism $f : X \to Y$ is an epimorphism if and only if it is surjective.
\end{exer}


\begin{exer}[\cref{sol:sections_in_set_injective}]\label{exer:sections_in_set_injective}
  In the category of sets, show that if $(s,r)$ is a section-retraction pair, then the section $s$ is injective.
  Hint: you can use \cref{ex:mono-inj}.
\end{exer}
\begin{exer}
  In the category of sets, show that if $(s,r)$ is a section-retraction pair, then the retraction $r$ is surjective.
    Hint: you can use \cref{ex:epi-surj}.
\end{exer}

\begin{exer}[\cref{sol:iso_to_monoepi}]\label{exer:iso_to_monoepi}
  Show that any isomorphism $f : a\cong b$ (in some arbitrary category $\CC$) is both a monomorphism and an epimorphism.
\end{exer}

\begin{exer}[\cref{sol:counterexample_monoepi_not_iso}]\label{exer:counterexample_monoepi_not_iso}
  Show that the converse of \cref{exer:iso_to_monoepi} does not hold in general, i.e. give an example of a category where there exists a morphism which is both an epi- and a monomorphism, but which is not an isomorphism.

Hint: Consider a preordered set.
\end{exer}

\begin{exer} Let $\mathcal{G}_1$ (resp. $\mathcal{G}_2$ and $\mathcal{G}_3$) be the category generated by the following graph:
\[
\begin{tikzcd}
& w & \\
x & & z \arrow[lu] \\
& y \arrow[lu] \arrow[ru] &
\end{tikzcd}
\]
resp.
\[
\begin{tikzcd}
& w & \\
x & & z \arrow[lu, bend left] \arrow[lu,bend right] \\
& y \arrow[lu] \arrow[ru] &
\end{tikzcd}
\]
resp.
\[
\begin{tikzcd}
& w \arrow[rd,bend left] & \\
x & & z \arrow[lu, bend left] \\
& y \arrow[lu] \arrow[ru] &
\end{tikzcd}
\]

Describe the mono- and epimorphisms in these categories.
\end{exer}

\begin{exer} Describe the monomorphisms, epimorphisms and isomorphisms in the category generated by the following graph:
\[
\begin{tikzcd}
x \arrow[r] & y
\end{tikzcd}
\]
\end{exer}